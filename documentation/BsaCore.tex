\documentclass[11pt]{article}

\usepackage{subfigure}
\usepackage[usenames]{color}
\usepackage{listings}
\lstset{basicstyle=\scriptsize, formfeed=\pagebreak}


\usepackage{lcls-article}

\lclsfancyheader{tidair}

\lclsdocumenttype{ICD}
\lclsdocumentnumber{}
\systemname{}

\newcommand{\bsac}{BsaCore}
\newcommand{\bsa} {BSA}
\newcommand{\slac}{SLAC}
\newcommand{\EDEF}{EDEF}
\newcommand{\cstl}[1]{{\lstinline+#1+}}

\newcommand{\cod}[1]{{\tt#1}}

\newcommand{\note}[1]{
	\begin{description}
		\item[NOTE] #1
	\end{description}
}

\newcommand{\example}[1]{
	\begin{description}
		\item[EXAMPLE] #1
	\end{description}
}

\newcounter{figs}

\newcommand{\fig}[3][0]{
\refstepcounter{figs}
\hfill\resizebox{#2}{!}{
        \rotatebox{#1}{\includegraphics{#3}}}\hspace*{\fill}
%\hfill\resizebox{#2}{!}{
%        \rotatebox{#1}{\includegraphics{#2}}}\hspace*{\fill}
}

\title{LCLS Beam-Synchronous Acquisition Core Software}
\author{Till Straumann}
\date{July 10, 2018}
% \addreviewer{who}{what}

\summary{%
The \acf{\bsac} software filters time-stamped data based on criteria set forth by
{\em event definitions} and forwards filtered and decimated data to a back-end/sink
for further processing. Data acquired by the \bsac facility are synchronous across
multiple computers.%
}

\renewcommand{\revnum}{DRAFT-0-1}

\addchangehistory{000}{July 10, 2018}{All}{DRAFT}

\currentrevision{0}

%
\keywords{\bsa, \bsac}


\begin{document}
\maketitle
\section{Introduction}
The \slac{} ``Beam-Synchronous Acquisition'' (\bsa) facility enables users to acquire
readings across the entire machine in a synchronous fashion for a well-defined (albeit
limited) number of beam pulses. The set of beam pulses on which data shall be acquired
is called an ``Event Definition'' or ``{\EDEF}''. Only a small number (the exact value differs
between LCLS-1 and LCLS-2) of independent \EDEF{}s is available and users have to
coordinate access to this limited resource. The exact selection criteria (such as
beam rate, time-slot etc.) for a particular beam-pulse to be included in an \EDEF{} is beyond
the scope of this document.

For each beam-pulse a bit-mask is broadcast over the timing-system (together with time-stamp
and other information). Each bit represents one \EDEF{} and communicates whether data acquired
for the respective beam-pulse shall or shall not be stored in the \bsa{} facility. An \EDEF{} 
with its correspoinding bit set to `1' is called ``active''. Thus,
on every beam-pulse data may be stored for zero, one or more active \EDEF{}s. Therefore, the data
store in \bsa{} may be visualized as a two-dimensional array of buffers with each row
representing a data source and each column representing an \EDEF. When a new data item
arrives it is dropped into all the columns that are active within the row corresponding
to the particular data source.

An additional feature of the \EDEF{}s is that data may be {\em averaged}. In fact, the 
timing system broadcasts additional bitmasks (along with the ``active'' mask) to support
averaging as well as hints for how data with abnormal status shall be handled by \bsa.

Once all the data requested by an \EDEF{} are stored in \bsa{} buffers they can be downloaded
by the user (usually via EPICS).

\begin{figure}
\fig{0.5\textwidth}{fig_blkdiag.eps}
\label{fig:bdiag}
\caption{\bsac{} with APIs and surrounding modules.}
\end{figure}


The \bsac{} software implements the functionality described above and has three
interfaces (see fig.~\ref{fig:bdiag}):
\begin{enumerate}
\item with the {\em timing system} in order to receive the \bsa-related bit-masks as well as
      the time-stamp.
\item with the {\em data source} from which \bsa{} receives time-stamped data items.
\item with one (or more) {\em data sinks} that consume the filtered \bsa{} data.
\end{enumerate}

The legacy implementation of \bsa{} was based on EPICS record-processing and had stringent
real-time requirements which were often difficult to meet. This led to data loss and 
inconsistencies with data sets across multiple systems no longer being synchronous.

The \bsac{} implementation discussed in this document aims at improving the behaviour by
introducing various levels of buffering that allow for relaxing the real-time requirements
significantly. Multi-threaded parallelism improves throughput on multi-core CPUs.

Note that in the context of lcls2 the \bsac{} software discussed here is {\em only} 
responsible for {\em slow} data sources which can be processed in software. \bsa{} for
high-performance systems (HPS) which is delegated largely to FPGA firmware is handled
by a {\em different} software module, not \bsac{}.

\section{Requirements}
The following list of requirements/design-goals was defined for \bsac:
\begin{description}
\item[Modularization] The legacy \bsa{} implementation was tightly coupled with the timing
   system (which in-turn was tied to the support of a specific hardware). For obvious reasons
   this monolithic approach is problematic and a new \bsa{} implementation should become an
   independent module with well-defined interfaces.
\item[Support for LCLS-i as well as LCLS-ii] The \bsa{} API as well as its implementation should
   be able to support LCLS-i as well as (slow) LCLS-ii devices and timing.
\item[Backwards Compatibility] The design of \bsa{} should allow for an easy migration path for
   existing (LCLS-i) applications. Ideally, it should be possible to port an IOC application 
   {\em without} making any changes, i.e., it should be sufficient to link against new libraries%
\footnote{This approach would, however, require a minimal amount of EPICS record-processing in real
time; with small modifications to the application such record-processing can be avoided.}.
\item[\bsac{} Software EPICS agnostic] The \bsac{} software should be a stand-alone component and
   avoid the use of EPICS records.
\item[Reduce RT-Requirements] The \bsac{} software design should reduce real-time processing
   requirements as much as possible.
\item[C-Language API] The \bsac{} APIs shall use the C-language.
\end{description}

\section{API}
Three interfaces are defined for the \bsac{} (see fig.~\ref{fig:bdiag}):
\begin{description}
\item[Timing System] which notifies the core of the arrival of a new timing pattern (time-stamp and
     \bsa{}-relevant bit-masks).
\item[Data Source] which notifies the core of the arrival of a new, {\em time-stamped} data item.
     The data may be discarded or stored for one or more \EDEF{}s, depending on the flags contained
     in the pattern with a time-stamp that matches the data time-stamp.
\item[Data Sink] is notified by \bsac{} by means of a user-defined callback that filtered and averaged
     \bsa{} data are available for consumption.
\end{description}
These APIs shall now be described in more detail.

\subsection{Timing API}
\bsac{} defines an indirect method of registering a callback function with the timing system.
The timing system's API defines a callback which is executed on arrival of each fiducial:
\begin{lstlisting}
/**
 * BSA Timing Pattern data
 */
typedef struct BsaTimingData
{
    TimingPulseId       pulseId;            /**< 64bit pulseId  */
    epicsTimeStamp      timeStamp;          /**< TimeStamp for this BSA timing data */
    uint64_t            edefInitMask;       /**< EDEF initialized mask  */
    uint64_t            edefActiveMask;     /**< EDEF active mask */
    uint64_t            edefAvgDoneMask;    /**< EDEF average-done mask */
    uint64_t            edefAllDoneMask;    /**< EDEF all-done mask */
    uint64_t            edefUpdateMask;     /**< EDEF update mask   */
    uint64_t            edefMinorMask;      /**< EDEF minor severity mask   */
    uint64_t            edefMajorMask;      /**< EDEF major severity mask   */
}   BsaTimingData;

/**
 * BsaTimingCallbacks get called w/ 2 parameters
 *   - pUserPvt is any pointer the callback client needs to establish context
 *   - pNewPattern is a pointer to the new BSA timing data
 *
 * Timing services must guarantee the BSA timing pattern data in this structure is all
 * from the same beam pulse and does not change before the callback returns.
 */
typedef void (*BsaTimingCallback)( void * pUserPvt, const BsaTimingData * pNewPattern );

/**
 * RegisterBsaTimingCallback is called by the BSA client to register a callback function
 * for new BsaTimingData.
 *
 * The pUserPvt pointer can be used to establish context or set to 0 if not needed.
 *
 * Timing services must support this RegisterBsaTimingCallback() function and call
 * the callback function once for each new BsaTimingData to be compliant w/ this
 * timing BSA API.
 *
 * The Timing service may support more than one BSA client, but is allowed to refuse
 * attempts to register multiple BSA callbacks.
 *
 * Each timing service should provide it's timing BSA code using a unique library name
 * so we can have EPICS IOC's that build applications for multiple timing service types.
 */
extern int RegisterBsaTimingCallback( BsaTimingCallback callback, void * pUserPvt );
\end{lstlisting}
and \bsac{} provides an implementation of this callback which is not directly exposed, however.
Instead, the user must connect the callback to the timing system by calling
\begin{lstlisting}
/**
  * Obtain the timing callback function provided by BSA
  */
int
BSA_TimingCallbackRegister ( int (*registrar) (BsaTimingCallback, void *) ) ;
\end{lstlisting}
where usually \cod{RegisterBsaTimingCallback} (note that this function is exported by
the {\em timing system} whereas \cod{BSA\_TimingCallbackRegister} is provided by \bsac())
is passed as the \cod{registrar} argument:
\begin{lstlisting}
#include <bsaApi.h>

...
status = BSA_TimingCallbackRegister( RegisterBsaTimingCallback );
...
\end{lstlisting}
This way of indirect registration ensures that a valid ``context'' is always
passed to \cod{RegisterBsaTimingCallback}.
\subsection{Channels}
\bsa{}-sources or -sinks are associated with {\em channels}. A \bsa{} channel
identifies a particular data source that is acquired into \bsa. A channel
corresponds to a row in the two-dimensional array mentioned in the introduction
(the columns of this array represent \EDEF{}s). Channels are identified by {\em IDs}
which are arbitrary, user-defined strings (names). An EPICS {\em PV} name is customarily
used as an ID.

The following entry points are defined for creating and identifying channels:
\begin{lstlisting}
/**
  * Create or find a BsaChannel. If a channel with the given ID already
  * exists then a handle for the existing channel is returned. Otherwise
  * the handle for a newly created channel is returned.
  */
BsaChannel BSA_CreateChannel(const char *id);

/**
  * Look up a BsaChannel. If a channel with the given ID already
  * exists then a handle for the existing channel is returned.
  * Otherwise NULL is returned.
  */
BsaChannel BSA_FindChannel(const char *id);

/**
  * Get the ID of a channel.
  */
const char *BSA_GetChannelID(BsaChannel channel);
\end{lstlisting}

\subsection{Data Source API}
Every time a data source (associated with a \bsa{} channel) produces fresh
data it must notify the \bsac{} by calling \cod{BSA\_StoreData()}, providing
a timestamp as well as status information:
\begin{lstlisting}
/**
  * Returns 0 on success, nonzero on error.
  */
int
BSA_StoreData(
    BsaChannel     bsaChannel,
    epicsTimeStamp timeStamp,
    double         value,
    BsaStat        status,
    BsaSevr        severity
);
\end{lstlisting}
Note that currently only IEEE double-precision floating point values are supported.

For convenience, \bsac{} exports a routine to determine status and severity based
on comparing a value to ``alarm limits''. This routine replicates the (unfortunately
non-public) algorithm used by EPICS' {\em AI-record}:
\begin{lstlisting}
typedef struct BsaAlarmLimitsStruct {
	double  lolo, low, high, hihi;
	double  lalm, hyst;
	BsaSevr llsv, lsv, hsv , hhsv;
} BsaAlarmLimitsStruct, *BsaAlarmLimits;

/**
  * This routine is intended to be fast; if
  * access to the BsaAlarmLimits needs to be
  * protected the caller can employ a spinlock
  * or mutex.
  * NOTE: 'laml' is modified by this routine.
  *       'status' and 'severity' are updated
  *       (i.e., they already must have valid
  *        content since the severity is only
  *        increased by this routine).
  */
void
BSA_CheckAlarms(double val, BsaAlarmLimits levels, BsaStat *status, BsaSevr *severity);
\end{lstlisting}

\subsection{Data Sink API}
A \bsac{} {\em sink} consumes averaged and decimated (``filtered'') \bsa{} data.
These data are aggregated with additional information such as time-stamp, the number
of averaged readings and status information.
The user registers a callback table with \bsac{} and is notified when certain
events occur:
\begin{itemize}
\item when a new acquisition is started the \cod{OnInit()} callback is executed
      and provided a time-stamp which identifies the pulse on which the acquisition
      starts.
\item when new results become available the \cod{OnResult()} callback is invoked
      from the context of a worker thread. Note that under some circumstances
      multiple results are aggregated and passed to this callback for sake of
      efficiency.
\item when an acquisition is terminated abnormally then \cod{OnAbort()} is invoked
      with additional status information.
\end{itemize}
A sink (callback table) is registered with
\begin{lstlisting}
/*
 * Register a sink with a BSA channel for a given EDEF index.
 *
 * 'maxResults' specifies how many results may be delivered
 * to a single call of 'OnResult'.
 *
 * Returns: status (0 == OK)
 */
int
BSA_AddSimpleSink(
    BsaChannel        bsaChannel,
    BsaEdef           edefIndex,
    BsaSimpleDataSink sink,
    void             *closure,
    unsigned          maxResults
);
\end{lstlisting}
A sink is added for a specific channel and a specific \EDEF{}, i.e.,
to an individual ``cell'' in the two-dimensional matrix mentioned in the introduction.

Note that multiple sinks can be attached to each such ``cell'', i.e., the same
channel/\EDEF{} combination may have multiple sinks attached. \bsac{} iterates
over all the sinks when a relevant events needs to be dispatched.

The \cod{maxResults} argument specifies how many ``results'' \bsac{} {\em may}
accumulate before dispatching the \cod{OnResults()} callback. This merely
sets a limit but \bsac{} may actually deliver less than this limit. It is
thus mandatory that the user check the {\em actual} number of results
by inspecting the respective argument to the \cod{OnResults()} callback.

Aggregation of results is desirable for efficiency reasons.

If \cod{maxResults} is set to zero then the implementation is free to pick
a value.

The complete prototypes for the callbacks as well as type definitions of the
arguments to \cod{BSA\_AddSimpleSink()} can be found in the appendix or the
current version of the \cod{BsaApi.h} header.

\section{Implementation Notes}
In this section we discuss some features of \bsac{} which are not reflected
in the API but are related to the implementation.
\subsection{Backwards Compatibility and Latency Remarks}
\label{sec:latency}
One design goal for \bsac{} was backwards compatibility, i.e., making it easy
to switch from the legacy, EPICS-record based implementation to \bsac{} without
changing the upper-layer EPICS-record interface which interacts with the user.

However, since another design goal was relaxing the real-time requirements
any user application which implicitly or explicitly relies on \bsa{} processing
in real-time will most likely be impacted by a switch from legacy \bsa{} to
\bsac{}.

\begin{figure}
\fig{0.5\textwidth}{fig_impl.eps}
\label{fig:impl}
\caption{\bsac{} implementation details. Timing patterns and raw data samples
         are stored in ring-buffers for deferred processing. A thread pool
         filters and averages raw data and feeds the sink API via output
         buffers.}
\end{figure}

Relaxing the real-time constraints required the introduction of several
layers of {\em buffering} (see fig.~\ref{fig:impl}) within \bsac{} which
has the consequence that the timing at which \bsa{} results are published
are more relaxed than with the legacy implementation.
This relaxed timing is in addition to any
networking delays that a remote user (EPICS OPI) may observe. In particular,
there is no guaranteed latency from data being stored into \bsac{} to when
processed results are published to the sink(s).

E.g., the author has heard of users who are using a PV hosted on the EVG
to synchronize with termination of \bsa{}. I.e., it was assumed that
all (distributed!) \bsa{} data (for a particular \EDEF{}) are available
as soon as the aforementioned EVG PV indicated that the EVG had finished the
\EDEF{} in question. Obviously, this method relies on real-time 
processing of \bsa{} on all the IOCs.

A more robust algorithm for synchronizing \bsa{} buffers would
e.g, monitor the number of elements acquired and synchronize
the main thread:
\begin{lstlisting}
monitor:
  if ( number_of_elements >= desired ) then
    // read the actual data
    ca_get_bsa_channel( this_channel );
    if ( 0 == atomic_clear_flag( this_channel, &outstanding_mask ) then
      // last monitor gets the desired number of elements signals
      signal();
    end if
  endif

start_edef_and_wait:
  atomic_set_flag( all_channels_mask, &outstanding_mask );
  start_edef();
  wait_for_signal_or_timeout();
\end{lstlisting}

\subsection{\bsac{} Configuration and Initialization}
\bsac{} uses lazy initialization, i.e., the software is initialized
during the first execution of
\cod{BSA\_CreateChannel()} or \cod{BSA\_FindChannel()}.

Several internal parameters of the \bsac{} such as thread priorities
and buffer depths can be configured albeit only prior to initialization.
The respective calls have the prefix \cod{BSA\_Config} and are listed
in the \cod{BsaApi.h} header. Note, however, that it is not recommended
to change default configuration (maybe with exception of the thread
priorities).

\subsubsection{Sink Timeout}
A sink with $\cod{maxResults} >1$ may take a long time (if a \bsa{} runs
at a slow rate) or even forever (if a \bsa{} terminates before filling up
the last aggregate of results) to update. To prevent this from happening
\bsac{} periodically flushes all outstanding results and ensures that
users don't have to wait excessively. This global period can be adjusted
(\cod{BSA\_ConfigSetUpdateTimeoutSecs()}) but be aware that the default was
chosen to provide an optimal compromise between user-convenience and efficiency.
\appendix
\pagebreak
\section*{Appendix}

\subsection*{The \bsa{} API Header}
\label{app:api}
\lstinputlisting{../../src/BsaApi.h}
\subsection*{Integration with the EPICS \cod{event} module}
Porting the legacy \bsa{} (which historically has been part of
the \cod{event} module) to \bsac{} was possible with minimal
changes:
\begin{itemize}
\item \cod{bsaRecord} was modified to support the \bsa{}-related fields
      (\cod{VAL}, \cod{RMS}, \cod{PID}, ...) being {\em arrays} of length
      \cod{NELM}. A \cod{NORD} field was added which communicates the
      {\em actual} number of elements contained in the arrays (\cod{NELM}
      specifies the {\em maximum} of supported elements).

      These modifications are fully backwards compatible and the legacy
      \bsa{} implementation may still be used (\cod{NELM} is set to $1$
      in this case).

\item A new device-support module for \cod{bsaRecord} was created which
      implements a \bsac{} sink using the \cod{bsaRecord}'s \cod{NELM}
      field to define the \cod{maxResults} parameter. Thus, the problematic
      update rate of the chained \cod{compress} records can be reduced
      simply by increasing \cod{NELM} which causes the \cod{compress}
      records to be updated less frequently but with multiple new values
      at once. EPICS automatically takes care of this without any further
      ``special'' code. A value of $\cod{NELM} \approx 10$ has proven to
      reduce CPU load significantly.

      Note that while reducing \cod{NELM} also reduces the delay from data
      being stored into \bsac{} to results being published this delay cannot
      be completely eliminated (even with $\cod{NELM} = 1$) due to additional
      latency caused by internal buffering (see also above).

\item A new device-support module for the \cod{aoRecord} which forwards
      data into legacy \bsa{}. While \bsac{}-aware IOC applications are
      recommended to use the \bsa{} API for storing data into \bsac{}
      (\cod{BSA\_StoreData()}) this device-support module enables transparent
      porting of (\bsac{}-unaware) applications.
\end{itemize}
Using the aforementioned components is possible to switch existing applications
from legacy \bsa{} to \bsac{} simply by making small changes to the \bsa{}-related
database templates and linking the \bsac{} library.
\end{document}
